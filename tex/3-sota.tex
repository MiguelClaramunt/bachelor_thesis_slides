\begin{frame}{3. Antecedentes y tecnologías}

% El problema no se conceptualizó hasta la publicación de los trabajos de \citet{Cohen2011, Flew2012}, motivado por los avances en el procesado del lenguaje natural, bases de datos e \textit{information retrieval} de la época.

Se han desarrollado dos enfoques para abordar este problema:
\begin{itemize}
    \item \underline{\smash{Basado en patrones:}}  estilo o sintaxis, métricas de RRSS, minería de datos enfocada a emociones.
    \item \underline{\smash{Basado en evidencias.}} similaridad semántica entre \textit{claim} y evidencia.
\end{itemize} 

\vspace{2ex}

Trabajos relevantes: 
\begin{itemize}
    \item DeClarE \citep{Popat2018}
    \item HAN \citep{Ma2019}
    \item GET \citep{Xu2022}
\end{itemize}

\end{frame}

% \begin{frame}{Tecnologías}

% \textbf{DeClarE \citep{Popat2018}.} Basado en evidencias, utiliza redes neuronales convolucionales (CNN) para extraer características locales de las evidencias y contraevidencias; mientras que usa redes recurrentes (RNN) para capturar dependencias entre diferentes evidencias.


% \textbf{HAN \citep{Ma2019}.} Basado en patrones, extrae las \textit{features} mediante una \textit{Hierarchical Attention Network} (HAN) que extrae relaciones a nivel de palabra y a nivel de oración.

% \vspace{4ex}

% \textbf{GET \citep{Xu2022}.} Unifica los dos enfoques y propone una solución utilizando grafos, que permiten capturar dependencias semánticas a larga distancia.

% \end{frame}