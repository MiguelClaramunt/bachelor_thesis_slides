\begin{frame}{1.1. Introducción}

Las \textit{fake news} son piezas de texto con estilo pseudoperiodístico que tienen como objetivo desinformar a los lectores.

% \vspace{2ex}

% Estas `noticias' toman relevancia mundial en las elecciones presidenciales de E.E.U.U. en 2016, superando en \textit{engagement} a los medios tradicionales.

\vspace{2ex}

Motivos principales que incentivan la difusión de \textit{fake news}:
\begin{itemize}
    \item \underline{\smash{Económico:}} publicidad en la página web.
    % y el tráfico que generan estos sitios, pueden generar hasta 30.000 \$ mensuales.\footnote{\citep{Sydell2016}}
    \item \underline{\smash{Ideológico:}} apoyo de una agenda política.\footnote{\citep{AlcottGentzkow2017, Sydell2016}}
    % las \textit{fake news} consiguen enturbiar la opinión pública para apoyar una agenda política determinada.
\end{itemize}

\end{frame}

\begin{frame}{1.2. Motivación}

\textit{Fact checking}:
\begin{itemize}
    \item Exhaustivo, laborioso.
    \item Mayoritariamente manual.
    \item Grandes beneficios si se automatiza: \textit{LLMs}.
\end{itemize}

% Las redes sociales han transformado todo el proceso: formato, difusión, `viralidad', \textit{bots}, etc.

% \end{frame}

% \begin{frame}{Motivación}

% Las \textit{fake news}: 
% manipulan la percepción de la realidad, generando desconfianza y conflicto social.\footnote{\citep{CITSa}}

% % \vspace{2ex}

% El \textit{fact checking} es un proceso muy exhaustivo, laborioso y prácticamente manual, por lo se beneficiaría de la automatización.

\vspace{1ex}

% Los \textit{Large Language Models} (LLMs) permiten desarrollar aplicaciones para automatizar partes de este proceso.

% las siguientes aplicaciones: clasificador de noticias verdaderas/falsas, \textit{Information Retrieval}, resumen automático.

\vspace{2ex}

Otros beneficios: reducción de prejuicios en colectivos minorizados, mejora de la imagen del periodismo/periodista, etc.

\end{frame}

\begin{frame}{1.3. Objetivos}

Objetivo principal:
\begin{itemize}
    \item Desarrollar una solución que permita detectar noticias falsas.
    \begin{itemize}
        \item Clasificación a partir de características estilísticas.
        \item Estudio comparativo: cantidad/calidad de información, arquitectura/tamaño de los modelos. 
    \end{itemize}
\end{itemize}

\vspace{2ex}

Objetivos transversales:
\begin{itemize}
    \item Definición del término de \textit{fake news} y discutir las implicaciones en nuestro trabajo.
    \item Conocer en profundidad cómo funcionan los modelos y los sesgos implícitos.
    \item Aplicación de técnicas de \textit{Explanable AI} (XAI) para intentar entender el `razonamiento' de los modelos.
    % \item Conocer los sesgos de los modelos y cómo afectan la arquitectura y el aprendizaje en estos.
\end{itemize}

\end{frame}