\begin{frame}{2. Definiciones y área de estudio}

% Diferentes formulaciones del concepto aplicadas a noticias \citep{AlcottGentzkow2017,Lazer2018}.

% \vspace{2ex}

% \citet{Mourao2019} encuentran que la mayoría de información que se difunde no sigue la estructura de una noticia, por lo que hace falta reformular el concepto.

% \vspace{2ex}

% A partir del trabajo de , utilizamos la siguiente definición:

% \vspace{2ex}

\begin{quotation}
    \underline{Disinformation (Desinformación).} Contenido proposicional de signos que tergiversa el estado del mundo con la intención de engañar. --- \citep{Khan2021}   
\end{quotation}

\vspace{2ex}

Problema complejo, límite del area de aplicación a noticias:
\begin{itemize}
    \item \underline{\smash{Recopilables:}} gran disponibilidad de BBDD.
    \item \underline{\smash{Categorizables:}} aprendizaje supervisado.
\end{itemize}

% Ya que el problema es complejo, limitamos el área de aplicación a noticia, que son fácilmente recopilables y categorizables, ayudando al entrenamiento supervisado de los modelos.

\end{frame}